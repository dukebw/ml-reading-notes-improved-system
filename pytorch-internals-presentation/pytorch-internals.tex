\documentclass{beamer}
\usetheme{metropolis}
\usepackage{appendixnumberbeamer}

\usepackage{pgfplots}

\title{PyTorch Internals}
\date{\today}
\author{Brendan Duke}
\institute{University of Guelph}

\begin{document}

\maketitle

\begin{frame}{Table of contents}
  \setbeamertemplate{section in toc}[sections numbered]
  \tableofcontents[hideallsubsections]
\end{frame}

\section{Introduction}

\begin{frame}{nn.modules}
        \center{}
        \scalebox{.2}{\input{data/nn_modules.pdf_tex}}
\end{frame}

\begin{frame}{nn.modules.conv UML}
        \center{}
        \scalebox{.075}{\input{data/conv_classes.pdf_tex}}
\end{frame}

\begin{frame}[fragile]{Questions to answer}
        \begin{itemize}[<+- | alert@+>]
                \item How and where is the backprop algorithm implemented in
                        PyTorch?

                \item What is the full stack of software that runs when
                        training a two-layer ConvNet on CIFAR-10?
        \end{itemize}

        Blah~\cite{DBLP:journals/corr/RussakovskyDSKSMHKKBBF14}.
\end{frame}

\begin{frame}[allowframebreaks]{References}
        \bibliography{pytorch-internals}
        \bibliographystyle{apalike}
\end{frame}

\end{document}
